\documentclass[parskip=full,11pt]{scrartcl}
%\usepackage{pdfpages}
\usepackage[utf8]{inputenc}
\usepackage[T1]{fontenc}
\usepackage[german]{babel}
\usepackage[yyyymmdd]{datetime} 
\usepackage{hyperref}
\usepackage[toc, nonumberlist]{glossaries}
\usepackage{csquotes}
\hypersetup{
		pdftitle={Pflichtenheft},
		bookmarks=true,
}
\usepackage{fancyhdr}%<-------------to control headers and footers
\usepackage{tabularx}%<------------- simpler table management
\usepackage{float} 
\usepackage[a4paper,margin=1in,footskip=.25in]{geometry}
\fancyhf{}
\fancyfoot[C]{\thepage} %<----to get page number below text
\pagestyle{fancy} %<-------the page style itself

\title{Pflichtenheft}
\subtitle{Authorisierungsmanagement für eine virtuelle Forschungsumgebung für Geodaten}
\author{Alex\\Anastasia\\Atanas\\Dannie\\ Houra\\Sonya\\}
\date{22.11.17}

% define custom lists
\usepackage{enumitem}

% add glossary
\usepackage{glossaries}
\makeglossaries
\newglossaryentry{V-FOR-WaTer}
{
	name=V-FOR-WaTer,
	description={Die Virtuelle Forschungsumgebung für die Wasser- und Terrestrische Umweltforschung (V-FOR-WaTer)ist eine generische, virtuelle Forschungsumgebung für den gemeinsamen, systemischen Umgang mit Daten aus der Wasser- und Umweltforschung}, 
}
\newglossaryentry{Benutzer}
{
	name=Benutzer,
	description={definition} %todo
}
\newglossaryentry{Administrator}
{
	name={Administrator},
description={definition} %todo
}
\newglossaryentry{Ressourcenbesitzer}
{
	name={Ressourcenbesitzer},
	description={definition} %todo
}
\newglossaryentry{Löschen-Request}
{
	name={Löschen-Request},
	description={definition} %todo
}
\newglossaryentry{Web-Portal}
{
	name={Web-Portal},
	description={definition} %TODO bitte ergaenzen.
}
	\newglossaryentry{Zugriffsrechte}
{
	name={Zugriffsrechte},
	description=
	{
		Rechte den Inhalt einer Ressource zu lesen und auszuführen.
	}
}
\newglossaryentry{Besitz-Rechte}
{
	name={Besitz-Rechte},
	description=
	{
		Alle Rechte über die ein Ressourcenbesitzer verfügt. Besitz-Rechte implizieren Lese-Rechte.
	} 
}
\newglossaryentry{Admin-Rechte}
{
	name={Admin-Rechte},
	description=
	{
		%TODO definition ergänzen
	} 
}
\def\threedigits#1{%
  \ifnum#1<10 0\fi
  \ifnum#1<1 0\fi
  \number#1}
\begin{document}

\begin{titlepage}
	
	\begin{center}
	{\scshape\LARGE\bfseries Pflichtenheft \par}
	\vspace{1cm}
	{\scshape\Large Praxis der Softwareentwicklung\\}
	\vspace{1cm}
	{\scshape\Large Wintersemester 17/18\\}
	\vspace{3cm}
	{\huge\bfseries Authorisierungsmanagement für eine virtuelle Forschungsumgebung für Geodaten\par}
	\vspace{2cm}
	\vfill
	{\bfseries {\Large Autoren}:\par}
	{\Large Aleksandar Bachvarov}\\
	{\Large Anastasia}\\%TODO Nachname ergaenzen
	{\Large Atanas Dimitrov}\\
	{\Large Dannie}\\%TODO Nachname ergaenzen
	{\Large Houraalsadat Mortazavi Moshkenan}\\
	{\Large Sonya Voneva}\\
	\vfill
	{\large 26.11.17 \par}
	\end{center}
\end{titlepage}
\tableofcontents

\newpage
%\section{Einleitung}
\section{Zielbestimmung}
Das Produkt dient zum Authorisierungsmanagement des \gls{V-FOR-WaTer} Web-Portals. Dadurch können die in dem \gls{Web-Portal} registrierte \gls{Benutzer} Zugriffsanfragen für Ressourcen senden, Ressourcen nutzen und Ressourcen selbst erstellen. Dabei dient das Produkt auch zur Unterscheidung zwischen Benutzer, \gls{Ressourcenbesitzer} und \gls{Administrator}.

\subsection{Musskriterien}
Im Folgenden werden Kriterien aufgelistet, die auf jeden Fall umgesetzt werden.

\subsection*{Benutzer}
\begin{itemize}[itemsep=0pt]
\item Der Benutzer kann nach Ressourcen suchen.
\item Der Benutzer kann Ressourcen lesen, auf die er Lese-Rechte hat.
\item Der Benutzer kann ein Request dem Ressourcenbesitzer senden, um Lese-Rechte zu erwerben.
\item Der Benutzer bekommt Rückmeldung ob sein Request erfolgreich gesendet war.
\item Der Benutzer bekommt eine E-Mail-Benachrichtigung wenn seine Zugriffsanfrage genehmigt/abgelehnt wurde.
\item Der Benutzer kann seine eigenen Ressourcen erstellen. Damit wird er den Ressourcenbesitzer dieser Ressourcen.
\item Der Benutzer kann seinen Namen ändern.
\end{itemize}
 
\subsection*{Ressourcenbesitzer}
\begin{itemize}[itemsep=0pt]
\item Der Ressourcenbesitzer kann Lese-Rechte auf seine eigenen Ressourcen vergeben.
\item Der Ressourcenbesitzer kann Lese-Rechte auf seine eigenen Ressourcen einer Gruppe von Benutzern vergeben. %Gruppe in Glossar definieren
\item Der Ressourcenbesitzer kann freiwillig seine Besitz-Rechte mit anderen Benutzern teilen.
\item Der Ressourcenbesitzer kann ein \gls{Löschen-Request} für seine eigenen Ressourcen dem Admin senden.
\item Der Ressourcenbesitzer kann die E-Mail und Name vom Requst-Absender beim Request sehen.

\end{itemize}

\subsection*{Administrator}
\begin{itemize}[itemsep=0pt]
\item Der Admin kann Ressourcen löschen.
\item Der Admin kann Benutzer(vom Portal) entfernen.
\item Der Admin unterstützt die Datenbankverwaltung.
\item Der Admin kann Rechte auf Ressourcen beliebig vergeben (ohne selbst Ressourcenbesitzer zu sein).
\item Der Admin kann Ressourcenbesitzer ändern. 
\end{itemize}

\subsection{Wunschkriterien}
Im Folgenden werden Kriterien aufgelistet, die das Produkt umsetzen kann.
Im Verlauf des Entwurfs wird entschieden, welche der Kriterien  implementiert werden können.
\begin{itemize}[itemsep=0pt]
\item Benachrichtigung wenn eine Ressource gelöscht wird (nur an denen Benutzern, die Rechte darauf haben) 
\item Zugriffsanfrage für mehrere Ressourcen gleichzeitig senden
\item Verschiedene Möglichkeiten für Sortierung der Ressourcen
\item Der Benutzer kann ein Request für Admin-Rechte dem Admin senden.
\item Hilfeverweise für den Benutzer
\item Implementierung von Tokens zur Verifizierung von Rechten
\item Mehrmaliges Versagen eines Requests führt zur Benachrichtigung des Admins
\end{itemize}

\subsection{Abgrenzungskriterien}
Im Folgenden wird beschrieben, was das Produkt explizit nicht leisten soll.
\begin{itemize}[itemsep=0pt]
\item Das Produkt dient nicht zur Authentifizierung.
\item Das Produkt dient nicht zur Kommunikation zwischen Benutzern.
\item Das Produkt unterstützt keine Mobile-Version.
\item Die IDs von Benutzern sind nicht veränderbar.
\item Die E-Mail-Adressen von Benutzern sind nicht veränderbar.
\item Das Produkt steht nicht zur Verfügung für Benutzer ohne Account.
\end{itemize}


\section{Produkteinsatz}
Das Produkt wird in die Virtuelle Forschungsumgebung (VFU) für die Wasser-
und Terrestrische Umweltforschung (V-FOR-WaTer) im Rahmen des Netzwerks
Wasserforschung Baden-Württemberg eingesetzt. Die VFU legt ihre Schwerpunkte
auf die Datenhaltung und den Direkten Zugriff auf Analysewerkzeuge für Daten
aus der Wasser- und Umweltforschung. Das Produkt bezieht sich auf die
Rechteverwaltung für diese Daten.

\subsection{Anwendungsbereiche}
\begin{itemize}[itemsep=0pt]
\item Umweltforschungsbereich
\item Datenhaltung
\end{itemize}

\subsection{Zielgruppen}
\begin{itemize}[itemsep=0pt]
\item Administrator(en) der Webseite %(Mitarbeiter von V-FOR-WaTer)
\item Wissenschaftliche Mitarbeiter von V-FOR-WaTer
\item Externe Benutzer des Portals
\end{itemize}

\subsection{Betriebsbedingungen}
\begin{itemize}[itemsep=0pt]
\item Einsatz in einem Webportal mit einer Datenbank.
\item Das Produkt benötigt eine funktionierende Netzverbindung.
\item Der Betriebsdauer ist täglich 24 Stunden.
\end{itemize}


\section{Produktumgebung}
Das Produkt läuft auf einem Webportal.

\subsection{Software}
\begin{itemize}[itemsep=0pt]
\item SQLite – Datenbank
\item Moderne Webbrowser (Chrome, Safari, Edge, Firefox)
\item Betriebssystemunabhängig
\end{itemize}

\subsection{Hardware}
\begin{itemize}[itemsep=0pt]
\item Standartrechner
\end{itemize}


\section{Funktionale Anforderungen}
\subsection{Benutzerkontofunktionen}
\begin{enumerate}[label={\textbf{/F\protect\threedigits{\theenumi}0/}}, leftmargin=*]
\item \textit{bla}

\end{enumerate}

\subsubsection{Administratorfunktionen}
\begin{enumerate}[label={\textbf{/F\protect\threedigits{\theenumi}0/}}, leftmargin=*, resume]
\item \textit{bla}
\end{enumerate}

\section{Produktdaten}
\begin{enumerate}[label={\textbf{/D\protect\threedigits{\theenumi}0/}}, leftmargin=*]
\subsubsection{Personendaten}
\item \textit{Benutzerdaten:}
\begin{itemize}
   \item Benutzername
   
   \item Kennung:
   \begin{itemize}
     \item Benutzername
     \item Passwort
   \end{itemize}
 
   \item Persönliche Daten:
   \begin{itemize}
     \item Vorname
     \item Nachname
     \item Alter
     \item Geschlecht
     \item ID
     \item Institut 
   \end{itemize}
   
    \item Kontaktinformationen:
   \begin{itemize}
     \item Straße und Hausnummer
     \item Postleitzahl 
     \item Ort
     \item Land
     \item Fax
     \item Telefon
     \item E-Mail Adresse  
   \end{itemize}
   
    \item Sonstiges:
   \begin{itemize}
     \item Rechte
     \item Status (Administrator, Benutzer) 
     \item Letzte Anmeldung (Datum) 
     \item Registrierungsdatum (Datum)    
   \end{itemize}
 
\end{itemize}

\item \textit{Gruppendaten:}
\begin{itemize}
\item Administrator
\item Institut
\item Teilnehmer
 \end{itemize}
 
\subsubsection{Webportal Daten}
\item \textit{Datenliste?:}
\begin{itemize}
\item ID
\item Besitzer
\item Leser
 \end{itemize}
 
 \item \textit{Tools-liste??:}
\begin{itemize}
\item ID
\item Besitzer
\item Benutzer
 \end{itemize}

\end{enumerate}
\section{Nichtfunktionale Anforderungen}
\begin{enumerate}[label={\textbf{/NF\protect\threedigits{\theenumi}0/}}, leftmargin=*]
\item Eine Änderung von Rechten wird nach nächster Seitenaktualisierung sichtbar. Seitenaktualisierung geschieht automatisch alle X Sekunden.
\item Zur Erstellung eines Requests sind maximal X Schritte nötig.
\item Eine Änderung von Rechten führt nicht zur Veränderung von Ressourcen.
\item Eingabefelder, die Pflicht für den Benutzer sind, sollen mit einem Sternchen markiert werden.
\end{enumerate}



\section{Benutzerschnittstelle}

\section{Qualitätsbestimmungen}

\renewcommand{\arraystretch}{1.5}
\begin{table}[H]
  \begin{center}
    \begin{tabularx}{\textwidth}{X c c c c}
      \hline
      
      \textbf{{\large Produktivität}} & \textbf{{\large sehr gut}} & \textbf{{\large gut}} & \textbf{{\large normal} } &\textbf{{\large nicht relevant }}\\
      
      \hline      
      \multicolumn{5}{l}{\textbf{Funktionalität}}\\      
      \hline      
      Angemessenheit & x & x & x & x\\
	  Richtigkeit & x & x & x & x\\
	  Interoperabilität & x & x & x & x\\
	  Ordnungsmäßigkeit & x & x & x & x\\	
      Sicherheit & x & x & x & x\\	
		    
	  \hline	  
      \multicolumn{5}{l}{\textbf{Zuverlässigkeit}}\\     
      \hline
      Reife & x & x & x & x\\
	  Fehlertoleranz & x & x & x & x\\
	  Wiederherstellbarkeit & x & x & x & x\\
		
	  \hline	  	
	  \multicolumn{5}{l}{\textbf{Benutzbarkeit}}\\
      \hline
      Verständlichkeit & x & x & x & x\\
	  Erlernbarkeit & x & x & x & x\\
	  Bedienbarkeit & x & x & x & x\\
	  
	  \hline	  	
	  \multicolumn{5}{l}{\textbf{Effizienz}}\\
      \hline
      Zeitverhalten & x & x & x & x\\
	  Verbrauchsverhalten & x & x & x & x\\	
	  
	  \hline	  	
	  \multicolumn{5}{l}{\textbf{Änderbarkeit}}\\
      \hline
      Analysierbarkeit & x & x & x & x\\
	  Modifizierbarkeit & x & x & x & x\\
	  Stabilität & x & x & x & x\\
	  Prüfbarkeit & x & x & x & x\\
	  
	  \hline	  	
	  \multicolumn{5}{l}{\textbf{Benutzbarkeit}}\\
      \hline
      Anpassbarkeit & x & x & x & x\\
	  Installierbarkeit & x & x & x & x\\
	  Konformität & x & x & x & x\\
	  Austauschbarkeit & x & x & x & x\\
	  
	  \hline      			
    \end{tabularx}
  \end{center}
  
\end{table}
\renewcommand{\arraystretch}{1}
\section{Globale Testfälle und Testszenarien}

\newpage
\printglossary	
\end{document}
\grid
\grid
