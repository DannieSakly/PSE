\documentclass[parskip=full,11pt]{scrartcl}
%\usepackage{pdfpages}
\usepackage[utf8]{inputenc}
\usepackage[T1]{fontenc}
\usepackage[german]{babel}
\usepackage[yyyymmdd]{datetime} 
\usepackage{hyperref}
\usepackage[toc, nonumberlist]{glossaries}
\usepackage{csquotes}
\hypersetup{
		pdftitle={Pflichtenheft},
		bookmarks=true,
}
\usepackage{fancyhdr}%<-------------to control headers and footers
\usepackage[a4paper,margin=1in,footskip=.25in]{geometry}
\fancyhf{}
\fancyfoot[C]{\thepage} %<----to get page number below text
\pagestyle{fancy} %<-------the page style itself

\title{Pflichtenheft}
\subtitle{Autorisierungsmanagement für eine virtuelle Forschungsumgebung für Geodaten}
\author{Alex\\Anastasia\\Atanas\\Dannie\\ Houra\\Sonya\\}
\date{26.11.17}

% define custom lists
\usepackage{enumitem}

% add glossary
\usepackage{glossaries}
\makeglossaries
\newglossaryentry{V-FOR-WaTer}
{
	name=V-FOR-WaTer,
	description={definition} %todo
}
\newglossaryentry{Benutzer}
{
	name=Benutzer,
	description={definition} %todo
}
\newglossaryentry{Administrator}
{
	name={Administrator},
description={definition} %todo
}
\newglossaryentry{Ressourcenbesitzer}
{
	name={Ressourcenbesitzer},
	description={definition} %todo
}
\newglossaryentry{Löschen-Request}
{
	name={Löschen-Request},
	description={definition} %todo
}

\def\threedigits#1{%
  \ifnum#1<100 0\fi
  \ifnum#1<10 0\fi
  \number#1}
\begin{document}

\begin{titlepage}
	
	\begin{center}
	{\scshape\LARGE\bfseries Pflichtenheft \par}
	\vspace{1cm}
	{\scshape\Large Praxis der Softwareentwicklung\\}
	\vspace{1cm}
	{\scshape\Large Wintersemester 17/18\\}
	\vspace{3cm}
	{\huge\bfseries Autorisierungsmanagement für eine virtuelle Forschungsumgebung für Geodaten\par}
	\vspace{2cm}
	\vfill
	{\bfseries {\Large Autoren}:\par}
	{\Large Aleksandar Bachvarov}\\%TODO Nachname ergaenzen
	{\Large Anastasia}\\%TODO Nachname ergaenzen
	{\Large Atanas Dimitrov}\\%TODO Nachname ergaenzen
	{\Large Dannie}\\%TODO Nachname ergaenzen
	{\Large Houraalsadat Mortazavi Moshkenan}\\
	{\Large Sonya Voneva}\\%TODO Nachname ergaenzen
	\vfill
	{\large 26.11.17 \par}
	\end{center}
\end{titlepage}
\tableofcontents

\newpage
%Eineitung?
\section{Zielbestimmung}
Das Produkt dient  zum Autoriesierungsmanagement des  \gls{V-FOR-WaTer} Web-Portals.Dadurch können die in dem Webportal registierten \gls{Benutzer} ihre Profils verwalten, Ressourcen bearbeiten und zugreiffen.

\subsection{Musskriterien}
Im Folgenden werden Kriterien aufgelistet, die auf jeden Fall umgesetzt werden.

\subsection*{Benuzter}
\begin{itemize}[itemsep=0pt]
\item Unterscheidung zwischen \gls{Administrator} der Gruppe und Benutzer.
\item Unterscheidung zwischen \gls{Ressourcenbesitzer} und Benutzer.
 
\item Der Benutzer kann seine E-Mail-Adresse zu seinem Benutzeraccount hinzufügen.(?)
\item Der Benutzer kann nach Ressourcen suchen.
\item Der Benutzer kann zum Zugriff der Ressourcen ein Request dem Admistrator senden.
\item Der Benutzer kann seine eigene Ressourcen erstellen, besitzen und bearbeiten.
\item Der Benutzer bekommt eine E-Mail-Benachrichtigung wenn seine Zugriffsanfrage genehmigt/abgelehnt wurde.
\item Der Benutzer kann Ressourcen lesen, auf die er Rechte hat.
 
 \subsection*{Admin}
 \item Der Admin kann Ressourcen löschen (dafür braucht er kein Request zu senden).
 \item Der Admin kann Benutzer(vom Portal/von der Gruppe?) entfernen.
 \item Der Admin unterstützt die Datenbankverwaltung.
 
 
 \subsection*{Ressourcenbesitzer}
 \item Der Ressourcenbesitzer kann Rechte auf seine eigene Ressourcen vergeben.(Rechübergabe)
 \item Der Ressourcenbesitzer kann ein \gls{Löschen-Request} dem Admin senden.(?)
 
 \subsection*{Gruppe}(?)
\end{itemize}


\subsection{Wunschkriterien}
\begin{itemize}[itemsep=0pt]
\item Der Benutzer kann ein Profilbild zu seinem Benutzeraccount hinzufügen.(?)
\item Benachrichtigung wenn ein Ressource gelöscht wird
\item Zugriffsanfrage für mehrere Ressourcen gleichzeitig senden
\item Übersicht aller wissenschaftlichen Gruppen
\item Verschiedene Möglichkeiten für Sortierung der Ressourcen
\end{itemize}

\subsection{Abgerenzungskriterien}
\begin{itemize}[itemsep=0pt]
\item Das Produkt dient nicht zur Authentifikation.
\item Das Produkt dient nicht zur Kommunikation zwischen Benutzern.
\end{itemize}


\section{Produkteinsatz}

\subsection{Anwendungsbereiche}
\subsection{Zielgruppen}
\subsection{Betriebsbedingungen}


\section{Produktumgebung}
\subsection{Sofware}
\subsection{Hardware}


\section{Funktionale Anforderungen}
\subsection{Benutzerkontofunktionen}
\begin{enumerate}[label={\textbf{/F\protect\threedigits{\theenumi}}}, leftmargin=*]
\item \textit{bla}
\end{enumerate}

\subsubsection{Administratorfunktionen}
\begin{enumerate}[label={\textbf{/F\protect\threedigits{\theenumi}}}, leftmargin=*]
\item \textit{bla}
\end{enumerate}

\section{Produktdaten}
\begin{enumerate}[label={\textbf{/D\protect\threedigits{\theenumi}}}, leftmargin=*]
\item \textit{bla}
\end{enumerate}
\section{Nichtfunktionale Anforderungen}
\begin{enumerate}[label={\textbf{/NF\protect\threedigits{\theenumi}}}, leftmargin=*]
\item \textit{bla}
\end{enumerate}



\section{Benutzungsschnittstelle}

\section{Qualitätsbestimmungen}

\section{Globale Testfälle und Testszenarien}

\newpage
\printglossary	
\end{document}
\grid
\grid
