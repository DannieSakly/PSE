\documentclass[parskip=full,11pt]{scrartcl}
%\usepackage{pdfpages}
\usepackage[utf8]{inputenc}
\usepackage[T1]{fontenc}
\usepackage[german]{babel}
\usepackage[yyyymmdd]{datetime} 
\usepackage{hyperref}
\usepackage[toc, nonumberlist]{glossaries}
\usepackage{csquotes}
\hypersetup{
		pdftitle={Pflichtenheft},
		bookmarks=true,
}
\usepackage{fancyhdr}%<-------------to control headers and footers
\usepackage{tabularx}%<------------- simpler table management
\usepackage{float} 
\usepackage[a4paper,margin=1in,footskip=.25in]{geometry}
\fancyhf{}
\fancyfoot[C]{\thepage} %<----to get page number below text
\pagestyle{fancy} %<-------the page style itself

\title{Pflichtenheft}
\subtitle{Autorisierungsmanagement für eine virtuelle Forschungsumgebung für Geodaten}
\author{Alex\\Anastasia\\Atanas\\Dannie\\ Houra\\Sonya\\}
\date{26.11.17}

% define custom lists
\usepackage{enumitem}

% add glossary
\usepackage{glossaries}
\makeglossaries
\newglossaryentry{V-FOR-WaTer}
{
	name=V-FOR-WaTer,
	description={Die Virtuelle Forschungsumgebung für die Wasser- und Terrestrische Umweltforschung (V-FOR-WaTer)ist eine generische, virtuelle Forschungsumgebung für den gemeinsamen, systemischen Umgang mit Daten aus der Wasser- und Umweltforschung}, 
}
\newglossaryentry{Benutzer}
{
	name=Benutzer,
	description={definition} %todo
}
\newglossaryentry{Administrator}
{
	name={Administrator},
description={definition} %todo
}
\newglossaryentry{Ressourcenbesitzer}
{
	name={Ressourcenbesitzer},
	description={definition} %todo
}
\newglossaryentry{Löschen-Request}
{
	name={Löschen-Request},
	description={definition} %todo
}
\newglossaryentry{Web-portal}
{
	name={Web-portal},
	description={definition} %TODO bitte ergaenzen.
}
\def\threedigits#1{%
  \ifnum#1<100 0\fi
  \ifnum#1<10 0\fi
  \number#1}
\begin{document}

\begin{titlepage}
	
	\begin{center}
	{\scshape\LARGE\bfseries Pflichtenheft \par}
	\vspace{1cm}
	{\scshape\Large Praxis der Softwareentwicklung\\}
	\vspace{1cm}
	{\scshape\Large Wintersemester 17/18\\}
	\vspace{3cm}
	{\huge\bfseries Autorisierungsmanagement für eine virtuelle Forschungsumgebung für Geodaten\par}
	\vspace{2cm}
	\vfill
	{\bfseries {\Large Autoren}:\par}
	{\Large Aleksandar Bachvarov}\\
	{\Large Anastasia}\\%TODO Nachname ergaenzen
	{\Large Atanas Dimitrov}\\
	{\Large Dannie}\\%TODO Nachname ergaenzen
	{\Large Houraalsadat Mortazavi Moshkenan}\\
	{\Large Sonya Voneva}\\
	\vfill
	{\large 26.11.17 \par}
	\end{center}
\end{titlepage}
\tableofcontents

\newpage
%\section{Einleitung}
\section{Zielbestimmung}
Das Produkt dient zum Autoriesierungsmanagement des \gls{V-FOR-WaTer} Web-Portals.Dadurch können die in dem \gls{Web-Portal} registrierten \gls{Benutzer} ihre Profils verwalten, Ressourcen bearbeiten und zugreiffen.

\subsection{Musskriterien}
Im Folgenden werden Kriterien aufgelistet, die auf jeden Fall umgesetzt werden.

\subsection*{Benuzter}
\begin{itemize}[itemsep=0pt]
\item Unterscheidung zwischen \gls{Administrator} der Gruppe und Benutzer.
\item Unterscheidung zwischen \gls{Ressourcenbesitzer} und Benutzer.
 
\item Der Benutzer kann seine E-Mail-Adresse zu seinem Benutzeraccount hinzufügen.(?)
\item Der Benutzer kann nach Ressourcen suchen.
\item Der Benutzer kann zum Zugriff der Ressourcen ein Request dem Admistrator senden.
\item Der Benutzer kann seine eigene Ressourcen erstellen, besitzen und bearbeiten.
\item Der Benutzer bekommt eine E-Mail-Benachrichtigung wenn seine Zugriffsanfrage genehmigt/abgelehnt wurde.
\item Der Benutzer kann Ressourcen lesen, auf die er Rechte hat.
\end{itemize}
\subsection*{Administrator}
\begin{itemize}[itemsep=0pt]
\item Der Admin kann Ressourcen löschen (dafür braucht er kein Request zu senden).
\item Der Admin kann Benutzer(vom Portal/von der Gruppe?) entfernen.
\item Der Admin unterstützt die Datenbankverwaltung.
\end{itemize}
 
\subsection*{Ressourcenbesitzer}
\begin{itemize}[itemsep=0pt]
\item Der Ressourcenbesitzer kann Rechte auf seine eigene Ressourcen vergeben.(Rechübergabe)
\item Der Ressourcenbesitzer kann ein \gls{Löschen-Request} dem Admin senden.(?)
\end{itemize}

\subsection*{Gruppe}(?)
\begin{itemize}[itemsep=0pt]
\end{itemize}


\subsection{Wunschkriterien}
\begin{itemize}[itemsep=0pt]
\item Der Benutzer kann ein Profilbild zu seinem Benutzeraccount hinzufügen.(?)
\item Benachrichtigung wenn ein Ressource gelöscht wird
\item Zugriffsanfrage für mehrere Ressourcen gleichzeitig senden
\item Übersicht aller wissenschaftlichen Gruppen
\item Verschiedene Möglichkeiten für Sortierung der Ressourcen
\end{itemize}

\subsection{Abgerenzungskriterien}
\begin{itemize}[itemsep=0pt]
\item Das Produkt dient nicht zur Authentifikation.
\item Das Produkt dient nicht zur Kommunikation zwischen Benutzern.
\end{itemize}


\section{Produkteinsatz}

\subsection{Anwendungsbereiche}
\subsection{Zielgruppen}
\subsection{Betriebsbedingungen}


\section{Produktumgebung}
\subsection{Sofware}
\subsection{Hardware}


\section{Funktionale Anforderungen}
\subsection{Benutzerkontofunktionen}
\begin{enumerate}[label={\textbf{/F\protect\threedigits{\theenumi}/}}, leftmargin=*]
\item \textit{bla}

\end{enumerate}

\subsubsection{Administratorfunktionen}
\begin{enumerate}[label={\textbf{/F\protect\threedigits{\theenumi}/}}, leftmargin=*]
\item \textit{bla}
\end{enumerate}

\section{Produktdaten}
\begin{enumerate}[label={\textbf{/D\protect\threedigits{\theenumi}/}}, leftmargin=*]
\subsubsection{Personendaten}
\item \textit{Benutzerdaten:}
\begin{itemize}
   \item Benutzername
   
   \item Kennung:
   \begin{itemize}
     \item Benutzername
     \item Passwort
   \end{itemize}
 
   \item Persönliche Daten:
   \begin{itemize}
     \item Vorname
     \item Nachname
     \item Alter
     \item Geschlecht
     \item ID
     \item Institut 
   \end{itemize}
   
    \item Kontaktinformationen:
   \begin{itemize}
     \item Straße und Hausnummer
     \item Postleitzahl 
     \item Ort
     \item Land
     \item Fax
     \item Telefon
     \item E-Mail Adresse  
   \end{itemize}
   
    \item Sonstiges:
   \begin{itemize}
     \item Rechte
     \item Status (Administrator, Benutzer) 
     \item Letzte Anmeldung (Datum) 
     \item Registrierungsdatum (Datum)    
   \end{itemize}
 
\end{itemize}

\item \textit{Gruppendaten:}
\begin{itemize}
\item Administrator
\item Institut
\item Teilnehmer
 \end{itemize}
 
\subsubsection{Webportal Daten}
\item \textit{Datenliste?:}
\begin{itemize}
\item ID
\item Besitzer
\item Leser
 \end{itemize}
 
 \item \textit{Tools-liste??:}
\begin{itemize}
\item ID
\item Besitzer
\item Benutzer
 \end{itemize}

\end{enumerate}
\section{Nichtfunktionale Anforderungen}
\begin{enumerate}[label={\textbf{/NF\protect\threedigits{\theenumi}}}, leftmargin=*]
\item Eine Änderung von Rechten wird nach nächster Seitenaktualisierung sichtbar. Seitenaktualisierung geschieht automatisch alle X Sekunden.
\item Zur Erstellung eines Requests sind maximal X Schritte nötig.
\item Eine Änderung von Rechten führt nicht zur Veränderung von Ressourcen.
\item Eingabefelder, die Pflicht für den Benutzer sind, sollen mit einem Sternchen markiert werden.
\end{enumerate}



\section{Benutzungsschnittstelle}

\section{Qualitätsbestimmungen}

\renewcommand{\arraystretch}{1.5}
\begin{table}[H]
  \begin{center}
    \begin{tabularx}{\textwidth}{X c c c c}
      \hline
      
      \textbf{{\large Produktivität}} & \textbf{{\large sehr gut}} & \textbf{{\large gut}} & \textbf{{\large normal} } &\textbf{{\large nicht relevant }}\\
      
      \hline      
      \multicolumn{5}{l}{\textbf{Funktionalität}}\\      
      \hline      
      Angemessenheit & x & x & x & x\\
	  Richtigkeit & x & x & x & x\\
	  Interoperabilität & x & x & x & x\\
	  Ordnungsmäßigkeit & x & x & x & x\\	
      Sicherheit & x & x & x & x\\	
		    
	  \hline	  
      \multicolumn{5}{l}{\textbf{Zuverlässigkeit}}\\     
      \hline
      Reife & x & x & x & x\\
	  Fehlertoleranz & x & x & x & x\\
	  Wiederherstellbarkeit & x & x & x & x\\
		
	  \hline	  	
	  \multicolumn{5}{l}{\textbf{Benutzbarkeit}}\\
      \hline
      Verständlichkeit & x & x & x & x\\
	  Erlernbarkeit & x & x & x & x\\
	  Bedienbarkeit & x & x & x & x\\
	  
	  \hline	  	
	  \multicolumn{5}{l}{\textbf{Effizienz}}\\
      \hline
      Zeitverhalten & x & x & x & x\\
	  Verbrauchsverhalten & x & x & x & x\\	
	  
	  \hline	  	
	  \multicolumn{5}{l}{\textbf{Änderbarkeit}}\\
      \hline
      Analysierbarkeit & x & x & x & x\\
	  Modifizierbarkeit & x & x & x & x\\
	  Stabilität & x & x & x & x\\
	  Prüfbarkeit & x & x & x & x\\
	  
	  \hline	  	
	  \multicolumn{5}{l}{\textbf{Benutzbarkeit}}\\
      \hline
      Anpassbarkeit & x & x & x & x\\
	  Installierbarkeit & x & x & x & x\\
	  Konformität & x & x & x & x\\
	  Austauschbarkeit & x & x & x & x\\
	  
	  \hline      			
    \end{tabularx}
  \end{center}
  
\end{table}
\renewcommand{\arraystretch}{1}
\section{Globale Testfälle und Testszenarien}

\newpage
\printglossary	
\end{document}
\grid
\grid
