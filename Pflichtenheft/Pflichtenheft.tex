\documentclass[parskip=full,11pt]{scrartcl}
%\usepackage{pdfpages}
\usepackage[utf8]{inputenc}
\usepackage[T1]{fontenc}
\usepackage[german]{babel}
\usepackage[yyyymmdd]{datetime} 
\usepackage{hyperref}
\usepackage[toc, nonumberlist, automake]{glossaries} %added automake option
\usepackage{graphicx}
\usepackage{csquotes}
\usepackage{xcolor}
\definecolor{shadecolor}{RGB}{220,220,220}
\hypersetup{
		pdftitle={Pflichtenheft},
		bookmarks=true,
}
\usepackage{fancyhdr}%<-------------to control headers and footers
\usepackage{tabularx}%<------------- simpler table management
\usepackage{float} 
\usepackage[a4paper,margin=1in,footskip=.25in]{geometry}
\fancyhf{}
\fancyfoot[C]{\thepage} %<----to get page number below text
\pagestyle{fancy} %<-------the page style itself

\title{Pflichtenheft}
\subtitle{Authorisierungsmanagement für eine virtuelle Forschungsumgebung für Geodaten}
\author{Alex\\Anastasia\\Atanas\\Dannie\\ Houra\\Sonya\\}
\date{22.11.17}

% define custom lists
\usepackage{enumitem}

% add glossary
\makeglossaries
\newglossaryentry{V-FOR-WaTer}
{
	name=V-FOR-WaTer,
	description={Die Virtuelle Forschungsumgebung für die Wasser- und Terrestrische Umweltforschung (V-FOR-WaTer) ist eine generische, virtuelle Forschungsumgebung für den gemeinsamen, systemischen Umgang mit Daten aus der Wasser- und Umweltforschung.}, 
}
\newglossaryentry{Benutzer}
{
	name=Benutzer,
	description={Eine Person, die das Portal benutzt. Somit verfügt sie unter anderem über Profilinformationen, die auf dem Datenbank gespeichert sind.} 
}
\newglossaryentry{Administrator}
{
	name={Administrator},
description={Ein Administrator ist der Verwalter des Systems, er unterstützt die Datenbankverwaltung und er hat im Vergleich zu Standardbenutzer erweiterte Rechte. Er spielt auch die Rolle der Moderator, damit er die Rechte zum Benutzer einräumen und entziehen kann. Der Administrator ist ein Mitarbeiter von V-FOR-WaTer.} 
}
\newglossaryentry{Ressourcenbesitzer}
{
	name={Ressourcenbesitzer},
	description={Ein Ressourcenbesitzer ist ein Benutzer der eigenen Ressourcen erstellt hat.} %todo
}
\newglossaryentry{Loeschen-Request}
{
	name={Loeschen-Request},
	description={Ein Löschen-Request ist eine vom Ressourcenbesitzer zum Administrator gesendeter Request. Dieses Request entspricht dem Wunsch, eine oder mehrere Ressourcen zu löschen und wird negativ oder positiv beantwortet.}  %Löschen muss so geschrieben werden, sonst -> Probleme 
}
\newglossaryentry{Web-Portal}
{
	name={Web-Portal},
	description={Ein Web-Portal ist ein Anwendungssystem, das sich durch die Integration von Anwendungen, Prozessen und Diensten auszeichnet. Ein Portal stellt seinem Benutzer verschiedene Funktionen zur Verfügung, wie beispielsweise Navigation und Benutzerverwaltung. Außerdem koordiniert es die Suche und die Präsentation von Informationen und soll die Sicherheit gewährleisten.} 
}
\newglossaryentry{Zugriffsrechte}
{
	name={Zugriffsrechte},
	description=
	{
		Rechte den Inhalt einer Ressource zu lesen und auszuführen.
	}
}
\newglossaryentry{Besitz-Rechte}
{
	name={Besitz-Rechte},
	description=
	{
		Alle Rechte über die ein Ressourcenbesitzer verfügt. Besitz-Rechte implizieren Zugriffsrechte.
	} 
}
\newglossaryentry{Administratorrechte}
{
	name={Administratorrechte},
	description=
	{
		%TODO definition ergänzen
	} 
}
\def\threedigits#1{%
  \ifnum#1<10 0\fi
  \ifnum#1<1 0\fi
  \number#1}
\begin{document}

\begin{titlepage}
	
	\begin{center}
	{\scshape\LARGE\bfseries Pflichtenheft \par}
	\vspace{1cm}
	{\scshape\Large Praxis der Softwareentwicklung\\}
	\vspace{1cm}
	{\scshape\Large Wintersemester 17/18\\}
	\vspace{3cm}
	{\huge\bfseries Authorisierungsmanagement für eine virtuelle Forschungsumgebung für Geodaten\par}
	\vspace{2cm}
	\vfill
	{\bfseries {\Large Autoren}:\par}
	{\Large Aleksandar Bachvarov}\\
	{\Large Anastasia Slobodyanik}\\
	{\Large Atanas Dimitrov}\\
	{\Large Khalil Sakly}\\
	{\Large Houraalsadat Mortazavi Moshkenan}\\
	{\Large Sonya Voneva}\\
	\vfill
	{\large 22.11.17 \par}
	\end{center}
\end{titlepage}
\tableofcontents

\newpage
%\section{Einleitung}
\section{Zielbestimmung}
Das Produkt dient zum Authorisierungsmanagement des \grqq{\gls{V-FOR-WaTer}}\grqq Web-Portals. Dadurch können die in dem \gls{Web-Portal} registrierte \gls{Benutzer} Zugriffsanfragen für Ressourcen senden, Ressourcen nutzen und Ressourcen selbst erstellen. Dabei dient das Produkt auch zur Unterscheidung zwischen Benutzer, \gls{Ressourcenbesitzer} und \gls{Administrator}.

\subsection{Musskriterien}
Im Folgenden werden Kriterien aufgelistet, die auf jeden Fall umgesetzt werden.

\subsection*{Benutzer}
\begin{itemize}[itemsep=0pt]
\item Der Benutzer kann Ressourcen lesen, auf die er \gls{Zugriffsrechte} hat.
\item Der Benutzer kann ein Zugriff-Request dem Ressourcenbesitzer senden, um Zugriffsrechte zu erwerben.
\item Der Benutzer bekommt Rückmeldung ob sein Request erfolgreich gesendet war.
\item Der Benutzer bekommt eine E-Mail-Benachrichtigung wenn seine Zugriffsanfrage genehmigt/abgelehnt wurde.
\item Der Benutzer kann seine eigenen Ressourcen erstellen. Damit wird er den Ressourcenbesitzer dieser Ressourcen.
\item Der Benutzer kann seinen Namen im Portal ändern.
\end{itemize}
 
\subsection*{Ressourcenbesitzer}
\begin{itemize}[itemsep=0pt]

\item Der Ressourcenbesitzer kann Zugriffsrechte auf seine eigenen Ressourcen auf Request oder freiwillig vergeben.
\item Der Ressourcenbesitzer kann freiwillig seine \gls{Besitz-Rechte} mit anderen Benutzern teilen.
\item Der Ressourcenbesitzer kann ein \gls{Loeschen-Request} für seine eigenen Ressourcen dem Administrator senden.
\item Der Ressourcenbesitzer kann den Name vom Requst-Absender beim Request sehen.

\end{itemize}

\subsection*{Administrator}
\begin{itemize}[itemsep=0pt]
\item Der Administrator kann Ressourcen löschen.
\item Der Administrator kann Benutzer(vom Portal) entfernen.
\item Der Administrator unterstützt die Datenbankverwaltung.
\item Der Administrator kann Zugriffsrechte auf Ressourcen beliebig vergeben (ohne selbst Ressourcenbesitzer zu sein).
\item Der Administrator kann Besitz-Rechte auf Ressourcen beliebig vergeben (ohne selbst Ressourcenbesitzer zu sein). 

\end{itemize}

\subsection{Wunschkriterien}
Im Folgenden werden Kriterien aufgelistet, die das Produkt umsetzen kann.
Im Verlauf des Entwurfs wird entschieden, welche der Kriterien  implementiert werden können.
\begin{itemize}[itemsep=0pt]
\item Benachrichtigung wenn eine Ressource gelöscht wird (nur an denen Benutzern, die Rechte darauf haben).
\item Zugriffsanfrage für mehrere Ressourcen gleichzeitig senden.
\item Der Benutzer kann ein Request für Administratorrechte dem Administrator senden.
\item Hilfeverweise für den Benutzer.
\item Implementierung von Tokens zur Verifizierung von Rechten.
\item Mehrmaliges Versagen eines Requests führt zur Benachrichtigung des Administrators.
\item Der Ressourcenbesitzer kann Zugriffsrechte auf seine eigenen Ressourcen einer Gruppe von Benutzern vergeben.
\end{itemize}

\subsection{Abgrenzungskriterien}
Im Folgenden wird beschrieben, was das Produkt explizit nicht leisten soll.
\begin{itemize}[itemsep=0pt]
\item Das Produkt dient nicht zur Authentifizierung.
\item Das Produkt dient nicht zur Kommunikation zwischen Benutzern.
\item Das Produkt unterstützt keine Mobile-Version.
\item Die IDs von Benutzern sind weder sichtbar, noch veränderbar.
\item Die E-Mail-Adressen von Benutzern sind nicht veränderbar.
\item Das Produkt steht nicht zur Verfügung für Benutzer ohne Account.
\end{itemize}


\section{Produkteinsatz}
Das Produkt wird in die Virtuelle Forschungsumgebung (VFU) für die Wasser-
und Terrestrische Umweltforschung (\grqq{\gls{V-FOR-WaTer}}\grqq) im Rahmen des Netzwerks
Wasserforschung Baden-Württemberg eingesetzt. Die VFU legt ihre Schwerpunkte
auf die Datenhaltung und den Direkten Zugriff auf Analysewerkzeuge für Daten
aus der Wasser- und Umweltforschung. Das Produkt bezieht sich auf die
Rechteverwaltung für diese Daten.

\subsection{Anwendungsbereiche}
\begin{itemize}[itemsep=0pt]
\item Umweltforschungsbereich
\item Datenhaltung
\end{itemize}

\subsection{Zielgruppen}
\begin{itemize}[itemsep=0pt]
\item Administrator(en) der Webseite %(Mitarbeiter von V-FOR-WaTer)
\item Wissenschaftliche Mitarbeiter von \grqq{\gls{V-FOR-WaTer}}\grqq
\item Externe Benutzer des Portals
\end{itemize}

\subsection{Betriebsbedingungen}
\begin{itemize}[itemsep=0pt]
\item Einsatz in einem Webportal mit einer Datenbank.
\item Das Produkt benötigt eine funktionierende Netzverbindung.
\item Der Betriebsdauer ist täglich 24 Stunden.
\end{itemize}


\section{Produktumgebung}
Das Produkt wird in die virtuelle Forschungsumgebung für Wasser- und Terrestrische Umweltforschung \grqq{\gls{V-FOR-WaTer}}\grqq integriert.\\
Das Produkt ist weitergehend unabhängig vom Betriebssystem, sofern folgende Produktumgebung vorhanden ist:

\subsection{Software}
\begin{itemize}[itemsep=0pt]
\item Server Seite:
	\begin{itemize}
	\item WebServer Apache 
	\item SQLite – Datenbank
	\end{itemize}
\item Client Seite:
	\begin{itemize}
	\item Moderne Webbrowser:
		\begin{itemize}
		\item Chrome
		\item Firefox
		\item Safari
		\item Microsoft Edge
		\end{itemize}
	
	\end{itemize}
\end{itemize}

\subsection{Hardware}
\begin{itemize}[itemsep=0pt]

	\item Server Seite:
	\begin{itemize}
	\item Netzwerkfähig
	\item Rechner, der die Ansprüche der o.g. Server-Software erfüllt.
	\end{itemize}
	\item Client Seite:
	\begin{itemize}
	\item Standardrechner
	\item Netzwerkverbindung
	\end{itemize}
\end{itemize}
\newpage


\section{Funktionale Anforderungen}
Im Folgenden werden die funktionale Anforderungen: sowohl Musskriterien als auch Wunschkriterien erläutert. Die Nummern optionaler Funktionalitäten, die sich aus den Wunschkriterien ergeben, sind \colorbox{shadecolor}{farblich gekennzeichnet}.
\subsection{Benutzerfunktionen}
\begin{enumerate}[label={\textbf{/F\protect\threedigits{\theenumi}0/}}, leftmargin=*]
\item \textbf{Profilübersicht:} \\ Der angemeldete Benutzer kann seine personenbezogene Daten (Name, Vorname, E-Mail-Adresse, ID) auf seiner Profilseite sehen.
\item \textbf{Datenänderung:} \\ Der angemeldete Benutzer kann seinen Namen ändern
\item \textbf{Ressourcenzugriff:} \\Der angemeldete Benutzer kann Ressource zugreifen, auf die er Zugriffsrechte hat.
\\ Von Ressourcen, auf die der Benutzer keine Rechte hat, sind nur die Meta-Daten sichtbar.
\item \textbf{Ressourcenerstellung:}\\ Der Benutzer kann neue Ressourcen hochladen, ihre Namen eingeben.
\item \textbf{Rechte auf Ressourcen anfordern:}\\ Der Benutzer kann Requests an den Ressourcenbesitzer senden, um die Rechte auf gewünschte Ressourcen zu erwerben.
\item \textbf{Benachrichtigung:}\\ Der Benutzer wird benachrichtigt durch eine E-Mail, wenn sein Request abgelehnt/genehmigt wird. 
\item \textbf{Requestsübersicht:}\\ Der angemeldete Benutzer kann seine abgesendete Requests auf seine Profilseite sehen.
\item \colorbox{shadecolor} {\textbf{Multiple Request:}}\\ Der Benutzer kann Zugriffs-Request für mehrere Ressourcen gleichzeitig senden.
\item \colorbox{shadecolor} {\textbf{Administratorrechte anfordern:}}\\ Der Benutzer kann ein Request an Administratorrechte senden, um Administratorrechte zu erwerben.
\end{enumerate}

\newpage
\subsection{Administratorfunktionen}
\begin{enumerate}[label={\textbf{/F\protect\threedigits{\theenumi}0/}}, leftmargin=*, resume]
\item \textbf{Bekommen von Löschen-Requeste}\\ Der Administrator bekommt Requeste von einem Ressourcenbesitzer zum Löschen von Ressourcen.\\\\
\textbf{Beschreibung:}\\
\begin{enumerate}[label=(\arabic*), leftmargin=*]
\item Nachdem der Ressourcenbesitzer ein Löschen-Request gesendet hat, werden alle Administratoren darüber durch eine Email und durch eine Nachricht im Portal benachrichtigt.\\
\item Die Nachricht enthält die Daten des Ressourcenbesitzers und den Namen der Ressource zum Löschen.\\ 
\item Der Administrator kann seine Entscheidung direkt von der Nachricht durch zwei Knöpfe \grqq{Ja}\grqq  oder \grqq{Nein}\grqq  treffen.
 
\end{enumerate}


\item \textbf{Löschen von Ressourcen}\\ Der Administrator darf die Ressourcen im Portal löschen.\\\\
\textbf{Beschreibung:}\\
\begin{enumerate}[label=(\arabic*), leftmargin=*]
\item Die Löschen kann entweder durch die in /F060/ beschriebene Weise oder auch ohne Request passieren.\\
\item Nach der Löschen der Ressource werden alle Ressourcenbesitzer(eventuell auch alle Benutzer mit Leserechte, bei Publicressourcen problematisch) durch eine Email darüber informiert. \\ 
\end{enumerate}

\item \textbf{Bekommen von Adminrequeste}\\ Der Administrator bekommt Requeste von einem Benutzer, der vom Admin verlangt, ein Admin zu werden.


\item \textbf{Rechte gewähren}\\ Der Administrator kann einem Benutzer Zugriff- oder Leserechte einräumen und wieder entziehen.\\\\
\textbf{Beschreibung:}\\
\begin{enumerate}[label=(\arabic*), leftmargin=*]
	\item Die Vergabe von Rechte kann entweder durch die in /F060/ beschriebene Weise oder auch ohne Request passieren.\\
	\item Nach der Vergabe von Rechte kann der Administrator sie wieder entziehen. \\ 
	\item Nach der Entnahme von Rechte kann der Administrator sie wieder vergeben. \\
	\end{enumerate}

\item \textbf{Benutzer löschen}\\ Der Administrator kann einen Benutzer löschen. \\\\
\textbf{Beschreibung:}\\
\begin{enumerate}[label=(\arabic*), leftmargin=*]
	\item Zusammen mit dem Account werden die personenbezogenen Daten von der Benutzer entfernt.\\
	\item Der Administrator kann der gelöschten Benutzers Account und Daten nicht wiederherstellen, sondern muss der Benutzer sich nochmal registrieren. \\ 
	\end{enumerate}

\item \colorbox{shadecolor} {\textbf{Benutzer blockieren}}\\ Der Administrator kann einen Benutzer blockieren, damit dieser sich nicht anmelden kann.\\\\
\textbf{Beschreibung:}\\
\begin{enumerate}[label=(\arabic*), leftmargin=*]
	\item Wenn der Benutzer blockiert ist, sind seinem Account und Daten nicht gelöscht, sondern kann er sich nicht mehr anmelden.\\
	\item Der Administrator kann der blockierende Benutzer wieder zum anmelden zulassen. \\ 
	\end{enumerate}


\item \colorbox{shadecolor} {\textbf{Benutzer suchen}}\\  Es kann ein Benutzer anhand von Vorname, Nachname durch den Administrator gesucht werden.\\\\
\textbf{Beschreibung:}\\
\begin{enumerate}[label=(\arabic*), leftmargin=*]
	\item Nach der Suche wird der gesuchte Benutzer dann angezeigt und er steht zur Verfügung.\\
	\end{enumerate}
\end{enumerate}

\subsection{Ressourcenbesitzerfunktionen}
\begin{enumerate}[label={\textbf{/F\protect\threedigits{\theenumi}0/}}, leftmargin=*, resume]
\item \textbf{Übersicht der Requests:}\\
Nach der Anmeldung wird dem Ressourcenbesitzer eine Liste von Requests, die von anderen Benutzern gesendet sind, angezeigt.  
\item \textbf {Übergabe der Zugriffsrechte:}\\ 
Der Besitzer kann Zugriffsrechte an anderen Benutzer vergeben.
\item \textbf {Request auf Zugriffsrechte ablehnen/genehmigen:}\\ 
Der Ressourcenbesitzer kann die gesendeten Requests, die von anderen Benutzern zur Anforderung einer Zugriffsrechte gesendet wurden, entweder ablehnen oder annehmen. In beiden Fälle werden der entsprechenden Benutzern benachrichtigt. Im Fall Genehmigung bekommt der Benutzer Die Zugriffsrechte auf  gewünschte Ressourcen.
\item \textbf{Übergabe der Besitz-Rechte:}\\
Der Ressourcenbesitzer kann seine Besitz-Rechte mit anderen Benutzern teilen.In diesem Fall haben die andere Benutzer dieselbe Rechte, die Ressourcenbesitzer hat. %todo
\item \textbf{Löschen-Request senden:}\\
Der Ressourcenbesitzer ist in der Lage zum Löschen seinen Ressourcen ein Request dem Administrator zu senden.
\item \colorbox{shadecolor} {\textbf{Löschbenachrichtigungen:}}\\
Beim Löschen einer Ressource werden allen Ressourcenbesitzer benachrihtigt.
\end{enumerate}


\section{Produktdaten}
\subsection{Benutzerdaten}
\begin{enumerate}[label={\textbf{/D\protect\threedigits{\theenumi}0/}}, leftmargin=*]
     		\item Vorname
     		\item Nachname
     		\item ID
     		\item E-Mail Adresse 
\end{enumerate}
     		 
\subsection{Sonstiges}
\begin{enumerate}[label={\textbf{/D\protect\threedigits{\theenumi}0/}}, leftmargin=*, resume]
		\item Ressourcenliste, die der Benutzer besitzt
        	\item Status (Administrator, Benutzer)     
\end{enumerate}


\subsection{Ressourcendaten}
\begin{enumerate}[label={\textbf{/D\protect\threedigits{\theenumi}0/}}, leftmargin=*, resume]
		\item Titel
		\item Erstellungsdatum
		\item Besitzer
        	\item Leser     
\end{enumerate}

\section{Nichtfunktionale Anforderungen}
\begin{enumerate}[label={\textbf{/NF\protect\threedigits{\theenumi}0/}}, leftmargin=*]
\item Eine Änderung von Rechten wird nach nächster Seitenaktualisierung sichtbar.
\item Zur Erstellung eines Requests sind maximal 5 Schritte nötig.
\item Zur Erstellung einer neuen Ressource sind maximal 5 Schritte nötig.
\item Eine Änderung von Rechten führt nicht zur Veränderung von Ressourcen.
\item Hilfeverweise %to do
\item Zeitverhalten - angemessen %to do
\end{enumerate}


\section{Systemmodelle}

\subsection{Szenarien}
\subsubsection*{Einfache BenutzerFunktionen}
Alice hat sich vor kurzem verheiratet und möchte darum ihren Familiennamen im Portal von \grqq{\gls{V-FOR-WaTer}}\grqq ändern. Sie loggt sich im System ein und sieht ihr Profil. Dann klickt sie auf dem Knopf \grqq{Name editieren}\grqq und tippt ihren neuen Namen ein. 

Nun möchte sie eine neue Ressource erstellen, darum klickt sie auf dem Knopf \grqq{Ressource  hinzufügen}\grqq. Später entscheidet sich Alice ihre neue Ressource X zu überprüfen. Sie sucht im Portal nach der Ressource X und findet sie. Weil sie der Besitzer ist, hat sie das Recht X zuzugreifen.\\

	\begin{center}
	\includegraphics[width=0.6\textwidth]{Szenario_1.png}
	\end{center}

\subsubsection*{Zugriff-Request senden}
Bob hat Interesse an einer Ressource X im Portal von \grqq{\gls{V-FOR-WaTer}}\grqq. Er probiert den Inhalt von X zuzugreifen. Leider hat er keine Zugriffsrechte darauf. Um solche Rechte zu bekommen, soll er ein Request am Besitzer von X senden, in diesem Fall - Alice. Bob wählt die Option \grqq{Zugriffsrechte anfragen}\grqq. 

Nach wenigen Sekunden bekommt Alice eine E-Mail-Benachrichtigung über das neue Request. Um zu entscheiden, ob sie die Zugriffsanfrage genehmigt, loggt sie sich im Portal ein. Dann sieht sie Bobs Anfrage und beschließt sie zu genehmigen. In kurzem bekommt Bob eine E-Mail mit der guten Neuigkeit. Nun kann er die Ressource zugreifen und ist glücklich.\\
	
	\begin{center}
	\includegraphics[width=1\textwidth]{Szenario_2.png}
	\end{center}

\newpage
\subsubsection*{Besitz-Rechte teilen}
Der Besitzer von Ressource Y, nämlich Bob, ist ein sehr beschäftigter Mann. Er hat keine Zeit und Lust jeden Tag die zahlreiche Zugriffsanfragen für Rechte auf Y zu beantworten. Deshalb hat er sich entschieden seine Besitz-Rechte mit drei Mitarbeitern von ihm zu teilen. \\
Er findet die Ressource Y und wählt die Option \grqq{Besitz-Rechte teilen}\grqq. Dann findet er die Personen durch ihre Namen. Auf diese Weise kann jeder von der neuentstandene Gruppe von Besitzern eine zukünftige Zugriffsanfrage genehmigen oder ablehnen.\\

	\begin{center}
	\includegraphics[width=0.7\textwidth]{Szenario_3bevor.png}
	
	\includegraphics[width=0.7\textwidth]{Szenario_3nach.png}
	\end{center}

\subsubsection*{Ressource editieren}
Alice hat ihre Meinung geändert und möchte jetzt etwas in Ressource X korrigieren. Da das Portal diese Option nicht anbietet, muss sie zuerst erneut die Ressource X erstellen. Dieses mal beinhaltet X aber auch die gemachte Korrektion. Danach sendet sie ein Löschen-Request am Administrator, damit er die alte Version von X zerstört.\\
	
	\begin{center}
	\includegraphics[width=1\textwidth]{Szenario_4.png}
	\end{center}
	
\subsection{Diagramme}

\subsection*{Funktionenübersichtsdiagramm}

	\begin{center}
	\includegraphics[width=0.95\textwidth]{Funktionen_uebersicht.png}
	\end{center}
	
\subsection*{Kontrollflussdiagramm \grqq{Ressource zugreifen}\grqq}

	\begin{center}
	\includegraphics[width=1\textwidth]{Kontrollfluss.png}
	\end{center}
	
\subsection*{Zustandsdiagramm \grqq{Zugriff-Request}\grqq}
Das Diagramm stellt die mögliche Zustände eines Zugriff-Requests dar, wenn ein Benutzer ohne Zugriffsrechte eine Ressource zugreifen will.
\\

\begin{center}
	\includegraphics[width=0.9\textwidth]{Zustandsdiagramm.png}
	\end{center}
	
\section{Qualitätsbestimmungen}

\renewcommand{\arraystretch}{1.5}
\begin{table}[H]
  \begin{center}
    \begin{tabularx}{\textwidth}{X c c c c}
      \hline
      
      \textbf{{\large Produktivität}} & \textbf{{\large sehr wichtig}} & \textbf{{\large wichtig}} & \textbf{{\large normal} } &\textbf{{\large nicht relevant }}\\
      
      \hline      
      \multicolumn{5}{l}{\textbf{Funktionalität}}\\      
      \hline      
      Angemessenheit &   & x &   &  \\
	  Richtigkeit & x &   &   &  \\
	  Interoperabilität & x &   &   &  \\
      Sicherheit &   & x &   &  \\	
		    
	  \hline	  
      \multicolumn{5}{l}{\textbf{Zuverlässigkeit}}\\     
      \hline
      Reife &   &   & x &  \\
	  Fehlertoleranz &   &   &   & x\\
	  Wiederherstellbarkeit &   &   &   & x\\
		
	  \hline	  	
	  \multicolumn{5}{l}{\textbf{Benutzbarkeit}}\\
      \hline
      Verständlichkeit & x &   &   &  \\
	  Erlernbarkeit &   & x &   &  \\
	  Bedienbarkeit & x &   &   &  \\
	  
	  \hline	  	
	  \multicolumn{5}{l}{\textbf{Effizienz}}\\
      \hline
      Zeitverhalten &   &   & x &  \\
	  Verbrauchsverhalten & x &   &   &  \\	
	  
	  \hline	  	
	  \multicolumn{5}{l}{\textbf{Änderbarkeit}}\\
      \hline
      Analysierbarkeit &   &   &   & x\\
	  Modifizierbarkeit & x &   &   &  \\
	  Stabilität &   & x &   &  \\
	  Prüfbarkeit &  & x &  & \\

	  
	  \hline	  	
	  \multicolumn{5}{l}{\textbf{Benutzbarkeit}}\\
      \hline
      Anpassbarkeit & x &  &  & \\
	  Installierbarkeit &  & x &  & \\
	  Konformität &  &  &  & x\\
	  Austauschbarkeit & x &  &  & \\
	  
	  \hline      			
    \end{tabularx}
  \end{center}
  
\end{table}
\renewcommand{\arraystretch}{1}

\section{Globale Testfälle}
\subsection{Benutzertestfälle}
\begin{enumerate}[label={\textbf{/T\protect\threedigits{\theenumi}0/}}, leftmargin=*]
\item Profilseite öffnen. (/F010/)
\item Namen ändern. (/F020/) 
\item Ressourcen öffnen. (/F030/)
\item Neue Ressource erstellen. (/F040/)
\item Request senden. (/F050/)
\item Benachrichtigungen kontrollieren. (/F060/) 
\item Requestliste ansehen. (/F070/)
\item Multiple Request senden. (/F080/)
\item Administratorrechte anfragen. (/F090/)
\end{enumerate}

\subsection{Ressourcenbesitzertestfälle}
\begin{enumerate}[label={\textbf{/T\protect\threedigits{\theenumi}0/}}, leftmargin=*, resume]
\item Liste von besessenen Ressourcen kontrollieren.  %todo
\item Zugriffsrechte an anderen Benutzer vergeben.
\item Requestsliste kontrollieren.
\item Request auf Zugriffsrechte ablehnen.
\item Request auf Zugriffsrechte genehmigen.
\item Besitzrechte an anderen Benutzer vergeben.
\item Request zum Löschen von Ressource absenden
\item Löschbenachrichtigung kontrollieren.  
\end{enumerate}

\subsection{Administratortestfälle}
\begin{enumerate}[label={\textbf{/T\protect\threedigits{\theenumi}0/}}, leftmargin=*, resume]
\item bu
\end{enumerate}
\newpage
\printglossary	
\glsaddall
\end{document}
\grid
\grid